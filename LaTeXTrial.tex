\documentclass{article}

\usepackage{Sweave}
\begin{document}
\Sconcordance{concordance:LaTeXTrial.tex:LaTeXTrial.Rnw:%
1 2 1 1 0 43 1 1 2 1 0 1 1 1 2 7 0 1 1 3 0 1 2 2 1}


% In LaTeX: the % sign is the comment character, not the # in R or shell scripts.

%remove asterisk to number titles
\section*{Main Title}

\subsection*{Second level title!}

\subsubsection*{Third level title}

Here is \textbf{bold face}.

Here is \textit{italic font}.

Here is \texttt{plain text}.


``Use two back ticks to start a quotation and two ingle quotes to end a quotation ''. 

\begin{itemize}
  \item first bullet point
  \item second bullet point
  \item third bullet point
\end{itemize}

\begin{enumerate}
  \item first bullet point
  \item second bullet point
  \item third bullet point
\end{enumerate}

Fortunately, most of what you do for writing in \LaTeX actually follows the rules for plain text!

\begin{verbatim}
Text here is literal, with no formatting signs.\\ Careful, there is a lso no margin control!
\end{verbatim}

\begin{quote}
This indents an entire paragraph for an extended quotations.
\end{quote}

%now let's fence for some r code
\begin{Schunk}
\begin{Sinput}
> x <- runif(10)
> y <- runif(10)
> #comments in R as always with a #
> print(x)
\end{Sinput}
\begin{Soutput}
 [1] 0.01096641 0.94403231 0.66538837 0.78180883 0.94423936 0.81089432
 [7] 0.14727531 0.36188152 0.91732565 0.40214386
\end{Soutput}
\begin{Sinput}
> plot(x,y)
\end{Sinput}
\end{Schunk}


\end{document}
